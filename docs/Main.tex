\chapter{Doing用户手册}

\section{Doing的命令行使用}

\subsection{安装}

要安装Doing,需要先安装Dotnet。因为Doing是基于Dotnet所编写的。在安装了Dotnet的情况下,一般可以使用以下命令(Windows下,可以通过可以使用git和dotnet的PowerShell来输入)安装:
\begin{lstlisting}
$ cd 你要安装的目录
$ git clone https://github.com/GOSCPS/doing.git
$ cd doing/doing
$ dotnet build --configuration Release
$ cd \bin\Release\net5.0
$ pwd
之后,doing应该位于当前目录。
你应该把它添加到你的系统的环境变量。
\end{lstlisting}


\subsection{使用}
要使用Doing,可以通过键入`doing --Help`来获取帮助。\newline{}
不过我们仍然有一些值得注意的点:\newline{}
\begin{itemize}
	\item 如果未设置,默认使用一个线程
	\item 如果未设置,默认使用`make.doing`作为构建文件
	\item 如果未设置,默认使用`Default`目标
	\item 如果未设置,默认碰到错误则退出
\end{itemize}



\section{Doing脚本的编写}
\subsection{编写Target}

Doing基于PowerShell,Doing里真正有实际作用的语句都通过PowerShell编写。如果你还不熟悉PowerShell,请先学习如何编写它。
那么,请看Hello World

\begin{lstlisting}[language=bash]
# 这是一个注释

@Target Default
	Write-Host "Hello World!"
@EndTarget
\end{lstlisting}
如你所见,这就是Doing的Hello World。把它保存到`make.doing`文件,然后直接调用doing,它应该输出`Hello World`。\newline{}

接下来让我们研究它的语法。
最简单的,注释:它使用\#作为注释字符。\#开头的任何行都被忽略。\newline{}

在第三行,我们定义了一个`Target`。`Target`是Doing构建的基本单位,也是最小单位。要定义一个Target很简单:
\begin{lstlisting}

# 注意,要使用@作为开头
@Target 你的Target名称

# Some Other

# 使用@EndTarget来结束Target
@EndTarget


# 注意:
# Doing可能会接受一些其他的Target的名称(如中文)
# 但我们仍然建议Target名称仅使用字母,下划线和数字的组合。
\end{lstlisting}
作为一个构建系统,没有依赖怎么能行呢?
\begin{lstlisting}

# 定义了一个Target
# 依赖于Dep

@Target Default : Dep

	Write-Host "From Default"
	
@EndTarget
\end{lstlisting}
看到了吧,要定义依赖很简单:在Target的名称之后附上`:`之后附上依赖列表就行了(多个依赖,使用空格分割即可)。

如果你直接运行上面的代码,会得到错误:
\begin{lstlisting}
*** Miss depend `Dep` in target `Default` !
\end{lstlisting}
看起来,我们需要一个名为Dep的Target。\newline{}
接下来,我们修改源文件:
\begin{lstlisting}
	
	# 定义了一个Target
	# 依赖于Dep
	
	@Target Default : Dep
	
	Write-Host "From Default"
	
	@EndTarget
	
	@Target Dep
	
	Write-Host "From Dep!"
	
	@EndTarget
\end{lstlisting}
之后运行doing,我们可以看到输出:
\begin{lstlisting}
From Dep!
From Default!
\end{lstlisting}
Doing正确地解析了他们之间的依赖关系。并且Default晚于其依赖Dep执行。


\subsection{编写Function}
除此之外,Doing还可以定义PowerShell函数。函数的命名规则同Target(PowerShell可能会做额外检查,所以需要也符合PowerShell的命名规则)。
\begin{lstlisting}

@Function 你的函数名称

# 你的函数体

@EndFunction

\end{lstlisting}
函数体已经处于函数作用域内,不需要额外定义:
\begin{lstlisting}
# 正确的函数定义
@Function ATestFunc
	
	Write-Host "Hello" $args[0]
	
@EndTarget

# 错误的函数定义
@Function AErrorFunc

# 这个函数将会什么都不做
function AErrorFunc{
	Write-Host "Hello" $args[0]
}

@EndTarget

\end{lstlisting}


\subsection{预处理}
Doing中,存在一些特别的注释,它可以修改Doing的行为。例如引用文件等,我们称他们为`预处理`。预处理以\#{}\#{}!开头。
目前版本中,Doing只有两个预处理选项:
\begin{lstlisting}
	
	# 通知编译器要引入文件
	# 但是不会直接复制到此处
	# 而是会解析引用的文件
	##! Import <要引用的文件名称>
	
	# 检查Doing版本是否大于等于此版本号
	# 如果小于,则会打印错误
	##! VersionRequired <版本号>
	

\end{lstlisting}



\subsection{Init和Main}
Doing中存在两个特别的Target:`Init`和`Main`。
这两个Target会比任何其他Target提前构建。但是Doing不保证`Init`和`Main`谁先构建。

\newline{}
每个文件都可以拥有这两个Target,这两个Target不能拥有依赖,其他Target也不能依赖他们。
他们的执行规则如下:

\begin{itemize}
	\item 如果要构建的Target所在的文件有`Main`,则构建一次Main
	\item 如果`Main`所在文件所定义的任何一个Target不需要构建,则不构建Target
	\item 任何加载过的文件的`Init`Target都会构建,与目标无关。
\end{itemize}
注意,`Main`和`Init`都只构建一次。



\section{Doing的Cmdlet参考}
Doing为PowerShell实现了一些有利于构建的Cmdlet。

%TODO Cmdlet简介









