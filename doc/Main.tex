\chapter{Doing 文档}
\section{Doing}
\subsection{Doing 简介}

Doing 是由 GOSCPS 所研发的一种新型构建系统。这套系统旨在实现构建扩展化,灵活化。这套系统使用C\#{}编写,并且可以轻易地被扩展。

\subsection{Doing 命令行使用}

doing使用C\#编写,因此通常可以在命令行直接键入doing.exe来使用。doing有两条基本命令:\newline{}
\begin{lstlisting}
	--help 			这一条用来显示帮助
	--version 		这一条用来显示版本
\end{lstlisting}
截至doing 1.0.0,doing提供了以下命令行设置:\newline{}
\begin{lstlisting}
	-D, --Define    Define global environment variables.
	
	-F, --File      (Default: build.doing) Set the doing file.Default is `build.doing`.
	
	-T, --Thread    (Default: -1) Define max thread count you want to use.
	
	--help          Display this help screen.
	
	--version       Display version information.
	
	value pos. 0    What target you want to build.
\end{lstlisting}
如果提供翻译,那么就是:\newline{}
\begin{lstlisting}
	-D, --Define    用法 -D Key1 Value1 Key2 Value2 ...
					用来定义全局环境变量
					Key和Value必须成对出现
	
	-F, --File      设置要构建的.doing文件
					如果没设置,默认使用build.doing
	
	-T, --Thread    设置要使用的线程数量
					默认-1
					如果小于等于0,则使用C#的(Environment.ProcessorCount)
					官方对此说法的原文是:
					The 32-bit signed integer that specifies the number of processors on the current machine. 
					There is no default. If the current machine contains 
					multiple processor groups,
					this property returns the number of 
					logical processors that are available for use by the common language runtime (CLR).
	
	--help          显示帮助页面
	
	--version       显示版本号
	
	value pos. 0  	你要构建的目标
					需要放在所有选项之前
					支持构建多个选项
\end{lstlisting}
下面是一个例子:
\begin{lstlisting}
$ doing.exe install update -D CC clang CXX clang++ -F build.doing -T 2
\end{lstlisting}
意为:在build.doing文件构建install和update目标,设置全局变量CC为clang,设置全局变量CXX为clang++,最大使用两个线程。


\subsection{Doing 的安装}
Doing官方已经提供了二进制分发:因其使用C\#编写,所以您可以在dotnet支持的平台运行它。不过GOSCPS官方支持的平台为:Windows10,Linux和MacOS。
\newline{}
如果你需要自己构建,也很简单。
\begin{lstlisting}
$ cd You Work Space
$ git clone https://github.com/GOSCPS/doing
$ cd doing/doing
$ dotnet build --configuration Release
$ cd ..
$ .\doing\bin\Release\net5.0\doing.exe install -D InstallPath 你想安装的目录
\end{lstlisting}
之后就能完成安装了。

\subsection{Doing Start}
Doing在语法上借鉴了makefile和\LaTeX{},这使得doing变得容易扩展而且容易编写。doing文件均以.doing作为结尾。
下面是一个doing版本的Hello World。
\newline{}
\begin{lstlisting}
target "default"
\"Print"{"Hello World"}
end
\end{lstlisting}
保存到build.doing。键入doing(.exe)即可运行。doing会自己找到文件并执行。Well,让我们解析一下这个简短的示例。
\newline{}

第一行:
\begin{lstlisting}
target "default"
\end{lstlisting}
定义了一个名为default的目标。目标名称需要双引号""括起来,虽然没有做强制要求,但建议使用符合正则表达式[a-zA-Z0-9\_]+的名称。
\newline{}

让我们看看第二行:
\begin{lstlisting}
\"Print"{"Hello World"}
\end{lstlisting}
这一行调用名为Print的宏。并把"Hello World"作为参数传入。doing的执行基本单位即是宏。注意宏的名称和参数也需要用双引号括住。

最后一行:
\begin{lstlisting}
end
\end{lstlisting}
与第一行对应。代表一个target到此为止。
\newline{}
\newline{}

现在让我们看看doing的正式语法定义:
\label{DoingBase}
\begin{lstlisting}
# doing正式语法定义
#以#开头的都是注释
#	会被忽略

# include会把include的文件原样插入到include的地点
include "File Name.doing"

# using会加载doing扩展
# doing扩展都位于doing可执行文件同级目录下的Expand目录
using "Expand Name"

# 定义一个目标
# target "目标的名称"
# 如果目标名称后接 `:` 字符,则字符`:`后的字符串组均视为这个目标的依赖
# 如果没有依赖可以写作:Target "Target Name"
# 构建目标时,doing会确保构建前依赖已经构建过
target "Target Name" : "Dep Target Name1" "Dep Target Name2" "Dep Target Name3"

# 宏
# \"宏名"{"宏参数"}
\"Macro Name"{"Macro Param"}

#结束target
#target不能嵌套
end

# 如果有宏不属于任何target
# 则会在解析语法时立刻执行

# 字符串解析(即用双引号括起来的字符串) 
# 都支持转义
# \n = 换行
# \\ = \
# \' = '
# \" = "
# \uXXXXX = 将XXXXX视作Unicode编码转义
# \t = tab

# doing内置有一个变量系统
# 适用于整个构建周期的,称为全局变量
# 适用于单个target的,称为局部变量

# 通常情况下,如果变量重名
# 那么局部变量会优先于全局变量被使用
\end{lstlisting}


\chapter{Doing 内建宏文档}
doing提供了一些内建宏,这些宏不需要using也可以使用。Doing内建了一个GetStringFromString();Doing的内置宏大用特用了这个函数,有必要学习一下:\newline{}
\label{GetStringFromStrin}
\begin{lstlisting}
输入一个字符串:
	如果是$开头,则去掉开头的$,然后把字符串视作变量名称。寻找变量然后返回。局部变量优先。
	
	如果是\$开头,则去掉开头的\,然后返回。
	
	如果是"开头,则视作字符串,解析转义返回,解析方式同`\ref{DoingBase}`中的字符串解析。
	
	如果是\"开头,则去掉开头的\,然后返回。
	
	如果不属于以上任何情况,则直接返回。
\end{lstlisting}


\section{Doing Control}
这个宏包提供了一些控制语句。

\subsection{Break}
这个宏会把参数打印。然后结束当前target的执行。

\subsection{Error}
这个宏会把参数用红色打印到Error。然后以错误的方式结束当前target的执行。这会导致构建失败。

\subsection{Pass}
这个宏什么都不干,忽略参数。

\subsection{Backup}
这个宏会把参数视作变量名称,并备份,然后移除原来的变量。如果要备份的变量重名,局部变量优先。

\subsection{Restore}
这个宏会把参数视作已经被备份的变量的名称,会恢复其变量,然后移除其备份。如果要恢复的变量重名,局部变量优先。

\subsection{Global}
这个宏会模拟不在target内执行宏。用法为:
\begin{lstlisting}
\"Global"{"Macro Name:Macro Param"}
\end{lstlisting}
如果有多个`:`,则分界为最左边那一个。
简单的例子:
\begin{lstlisting}
target "default"
\"Global"{"Def:A=B"}
end

# 全局变量A,值为B
\end{lstlisting}

\section{Doing Def}

\subsection{Def}
定义一个变量等于一个值。这个宏的用法如下
\begin{lstlisting}
\"Def"{"Var Name = Value"}
\end{lstlisting}
Var Name和Value都会在Trim后被使用。Value还会被丢入\ref{GetStringFromStrin}解析。如果Def位于target内,则变量会被定义为局部变量,否则为全局变量。
\newline{}
简短的示例:
\begin{lstlisting}
\"Def"{"ImAVar = A"}
\"Def"{"ImA = $ImAVar"}
\"Def"{"ImHW = \"Hello World\""}

#ImAVar=A
#ImA=A
#ImHW=Hello World
\end{lstlisting}

\subsection{Undef}
把参数作为变量名称,然后取消定义这个变量。如果Undef位于target内,则会取消为局部变量,否则为全局变量。
取消定义的变量可以再次被定义。


\section{Doing If}
Doing内置提供了一组逻辑处理宏。\textbf{这些宏只能在target内使用。}

\subsection{Ifdef}
将宏参数作为变量名称。然后寻离宏最近的endif,找如果局部变量或者全局变量\textbf{有}定义这个变量,则把endif替换为Pass宏。否则,把endif和这个宏到endif之间的所有宏都替换到Pass。
\newline{}
简单的例子:
\begin{lstlisting}
target "default"
\"Def"{"A=B"}

\"Ifdef"{"A"}
\"Print"{"Hello World"}
endif

end
# 将会输出Hello World
\end{lstlisting}

\subsection{Ifndef}
Ifdef的孪生兄弟,作用相反。将宏参数作为变量名称。然后寻离宏最近的endif,找如果局部变量或者全局变量\textbf{没有}定义这个变量,则把endif替换为Pass宏。否则,把endif和这个宏到endif之间的所有宏都替换到Pass。
\newline{}
简单的例子:
\begin{lstlisting}
target "default"

\"Ifndef"{"A"}
\"Print"{"Hello World"}
endif

end
# 将会输出Hello World
\end{lstlisting}

\subsection{Ifeq}
判断两个值是否相等,如果相等,则寻离宏最近的endif,把endif替换为Pass宏。否则,把endif和这个宏到endif之间的所有宏都替换到Pass。用法形如:
\begin{lstlisting}
\"Ifeq"{"Left==Right"}
\end{lstlisting}
Left和Right都会被Trim后丢尽\ref{GetStringFromStrin}。之后做比较。

简单的例子:
\begin{lstlisting}
target "default"

\"Def"{"A=B"}

\"Ifeq"{"$A==B"}
\"Print"{"Hello World"}
endif

end
# 将会输出Hello World
\end{lstlisting}

\subsection{Ifneq}
Ifeq的孪生兄弟,作用相反。判断两个值是否不相等,如果不相等,则寻离宏最近的endif,把endif替换为Pass宏。否则,把endif和这个宏到endif之间的所有宏都替换到Pass。用法形如:
\begin{lstlisting}
\"Ifneq"{"Left!=Right"}
\end{lstlisting}
Left和Right都会被Trim后丢尽\ref{GetStringFromStrin}。之后做比较。

简单的例子:
\begin{lstlisting}
target "default"
	
\"Def"{"A=B"}
	
\"Ifneq"{"$A!=A"}
\"Print"{"Hello World"}
endif
	
end
# 将会输出Hello World
\end{lstlisting}


\section{Doing Sh}
Doing Sh提供了一些执行Shell用的代码

\subsection{Sh}
这个宏会把参数当作shell命令执行。
\begin{lstlisting}
\"Sh"{"echo \"Hello World\""}
# 在Windows上会执行:	pwsh.exe /c "echo \"Hello World\""
# 在unix上会执行:	/bin/sh -d "echo \"Hello World\""
\end{lstlisting}

\subsection{ShWithVar}
同Sh,但是会把全局变量和局部变量都添加到shell运行环境。

\section{Doing String}
Doing String提供了一些宏用来操作字符串。  

\subsection{StringAppend}
追加字符串:
\begin{lstlisting}
\"StringAppend"{"Var Name+=Value"}
\end{lstlisting}
Var Name和Value都会被Trim。Var Name会被视作要追加字符串的变量名称。Value会被送进\ref{GetStringFromStrin}。
之后Var Name会被追加Value。如果局部变量和全局变量都有Var Name,则局部变量优先。


\section{Doing Os}
Doing Os提供了一些宏用来和操作系统交互。

\subsection{AddToPath}
这个宏接受一个字符串,会被送入\ref{GetStringFromStrin}。然后把结果追加到系统的Path环境变量。

\section{Doing Print}
提供了一些用来打印的宏:

\begin{lstlisting}
Print系列:
Print
PrintWarn
PrintError
PrintGood

Printv系列:
Print
PrintvWarn
PrintvError
PrintvGood
\end{lstlisting}
Warn会打印黄色字符并输出到StdErr。\newline{}
Error会打印红色字符并输出到StdErr。\newline{}
Good会打印黄色字符并输出到StdOut。\newline{}

Print系列会直接打印参数。Printv则会把参数送到\ref{GetStringFromStrin}处理之后再打印。


\chapter{Doing 编写宏}
要在Doing编写宏很简单。\newline{}
下面是小小的例子:
\begin{lstlisting}[language=C++]
/// <summary>
/// 什么都不干的Pass宏
/// </summary>
/// <param name="param">宏参数</param>
/// <param name="interpreter">解释器</param>
/// <returns>
/// 如果执行成功返回true,否则false。
/// </returns>
[Api.Macro("Pass")]
public bool PassMacro(string param, Build.Interpreter.Interpreter interpreter)
{
	return true;
}
\end{lstlisting}
TODO。



