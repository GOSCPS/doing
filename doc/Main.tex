\chapter{Doing 文档}
\section{Doing}
\subsection{Doing 简介}

Doing 是由 GOSCPS 所研发的一种新型构建系统。这套系统旨在实现构建扩展化,灵活化。这套系统使用C\#{}编写,并且可以轻易地被扩展。

\subsection{Doing 命令行使用}

doing使用C\#编写,因此通常可以在命令行直接键入doing.exe来使用。doing有两条基本命令:\newline{}
\begin{lstlisting}
	--help 			这一条用来显示帮助
	--version 		这一条用来显示版本
\end{lstlisting}
截至doing 1.0.0,doing提供了以下命令行设置:\newline{}
\begin{lstlisting}
	-D, --Define    Define global environment variables.
	
	-F, --File      (Default: build.doing) Set the doing file.Default is `build.doing`.
	
	-T, --Thread    (Default: -1) Define max thread count you want to use.
	
	--help          Display this help screen.
	
	--version       Display version information.
	
	value pos. 0    What target you want to build.
\end{lstlisting}
如果提供翻译,那么就是:\newline{}
\begin{lstlisting}
	-D, --Define    用法 -D Key1 Value1 Key2 Value2
					用来定义全局环境变量
					Key和Value必须成对出现
	
	-F, --File      设置要构建的.doing文件
					如果没设置,默认使用build.doing
	
	-T, --Thread    设置要使用的线程数量
					默认-1
					如果小于等于0,则使用C#的(Environment.ProcessorCount)
					官方对此说法的原文是:
					The 32-bit signed integer that specifies the number of processors on the current machine. 
					There is no default. If the current machine contains 
					multiple processor groups,
					this property returns the number of 
					logical processors that are available for use by the common language runtime (CLR).
	
	--help          显示帮助页面
	
	--version       显示版本号
	
	value pos. 0  	你要构建的目标
					需要放在所有选项之前
\end{lstlisting}
下面是一个例子:
\begin{lstlisting}
$ doing.exe install -D CC clang -F build.doing -T 2
\end{lstlisting}
意为:在build.doing文件构建install目标,使用clang作为变量CC,最大使用两个线程。


\subsection{Doing 的安装}
Doing官方已经提供了二进制分发:因其使用C\#编写,所以您可以在dotnet支持的平台运行它。不过GOSCPS官方支持的平台为:Windows10,Linux和MacOS。
\newline{}
如果你需要自己构建,也很简单。
\begin{lstlisting}
$ cd You Work Space
$ git clone https://github.com/GOSCPS/doing
$ cd doing/doing
$ dotnet build --configuration Release
$ cd ..
$ .\doing\bin\Release\net5.0\doing.exe install -D InstallPath 你想安装的目录
\end{lstlisting}
之后就能完成安装了。




